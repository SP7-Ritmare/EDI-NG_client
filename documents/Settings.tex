	\subsection{Settings}

\subsubsection{userInterfaceLanguage}

Labels can be defined in as many languages as required, by using the \textit{xml:lang} attribute.

The userInterfaceLanguage tag defines which \textit{xml:lang} value should be selected for labels and help tooltips.

\subsubsection{metadataLanguage}

Defines the language to be used when retrieving datasets from datasources.

\subsubsection{metadataEndpoint}
Defines the endpoint of the EDI Server instance that should be used to convert the metadata into its XML format.

\subsubsection{sparqlEndpoint}
\label{sparqlEndpoint}
Defines the default SparQL endpoint.

\subsubsection{requiresValidation}
Can be set to false (default is true), if you want the metadata to be sent even if they have some errors.

\subsubsection{baseDocument}
This is, as the name suggests, the base of the XML document to be generated: it is a CDATA and it must include the root element, along with any namespaces that need to be defined.

\subsubsection{xsltChain}
\label{xsltChain}

A chain of XSL Transformations can be specified to further process the XML generated.

\begin{quote}{\textbf{Warning}}
	
	It is the template author's responsibility to make sure they are correct.
\end{quote}

