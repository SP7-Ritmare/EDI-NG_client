	\subsection{datasources}
Datasources provide valid values for specific \textit{items}.

A collection of datasources: each datasource can be one of \textit{codelist}, \textit{sparql} or \textit{singleton}.

\subsubsection{sparql}
The most general type of datasource is a SparQL query.

It has two attributes:
\begin{center}
	\begin{tabular}{ | p{0.2\textwidth} | p{0.7\textwidth} | }
		\hline
		Attribute & Description \\ 
		\hline
		xml:id & unique id \\  
		\hline
		endpointType & reference to an existing (declared) endpointType \\
		\hline
	\end{tabular}
\end{center}

It requires one child tag named \textit{query}, specifying the SparQL query.
\\
\\
Optional child tag \textit{url} allows to override the \textit{sparqlEndpoint} (see \ref{sparqlEndpoint}) specified in the \textit{settings} tag.
\\
Query can include a \textit{\$search\_param} token, which, if found, will be given a value based, for example, on user text.

\subsubsection{codelist}
A codelist is a simplified version of a \textit{sparql} datasource, based on a pre-defined query, accessed via its URI, specified by the child tag \textit{uri}.
\\
\\
Optional child tag \textit{url} allows to override the \textit{sparqlEndpoint} (see \ref{sparqlEndpoint}) specified in the \textit{settings} tag.
\\

\subsubsection{singleton}
\label{singleton}
A singleton is a special stateful \textit{sparql} datasource guaranteed to have only a single instance, so that it can be used to keep some items aligned to some other item whenever the latter changes.

The item triggering said alignment is specified by the attribute \textit{triggerItem}.

Another datasource is always needed, for the singleton to work: the trigger item refers to a sparql or codelist datasource, whereas the dependent items are connected to it via the singleton, which will refresh and select a single row of the singleton dataset, which is linked, in turn, to the uri of the row selected by the trigger item.
\\
\\
Optional child tag \textit{url} allows to override the \textit{sparqlEndpoint} (see \ref{sparqlEndpoint}) specified in the \textit{settings} tag.
\\


